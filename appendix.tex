% 鈴木研究室 卒業論文付録テンプレート
% Version: 1.1
% Last Update: 2011/12/31 by Hidekazu Suzuki
% Contact to: hsuzuki@meijo-u.ac.jp
% Character code: UTF-8

\chapter{付録に掲載する内容例}\label{apdx:Example}

付録には本文の内容を補う情報,データなどを掲載するとよい.
例えば,以下のような内容が付録として適切である.

\begin{itemize}
\item プロトコル,シーケンス,パケットフォーマットなどの詳細情報
\item プログラムのソースコード
\item 実装の詳細な情報,インストール方法,実行方法など
\item 関連研究の詳細
\item 本文中に記載しなかった実験データおよび評価結果など
\item 研究の過程で得られた知見,修正したプログラムのバグなど
\end{itemize}


\section{ソースコードを掲載する方法}

ソースコードを掲載する場合は,下記の様に{\TeX}ファイルに直接コードを記述する方法と,外部のソースファイルを読み込む方法がある.
{\TeX}ファイルの可読性などを考慮すると,\subsecref{subsec:lstinputlisting}の読み込む方法を推奨する.


\subsection{{\TeX}ファイルに直接記述する場合}
\label{subsec:lstlisting}

ソースコードを{\TeX}ファイルに直接記述する場合は,\texttt{lstlisting}環境内にソースコードを記述する.
また,\texttt{lstlisting}環境のオプションとして,ソースコードのプログラミング言語\footnote{サポートされているプログラミング言語は,下記Webサイトを参照すること.\\\url{https://www.overleaf.com/learn/latex/Code_listing#Supported_languages}},キャプションおよび本文から参照するためのラベルを指定する.

\begin{code}
\begin{lstlisting}[language=プログラミング言語,caption=キャプション,label=src:ラベル]
ソースコードを記述
\end{lstlisting}
\end{code}%

素数を判定するPythonプログラムの記述例を\srcref{src:PrimeNumDescription}に示す.

\begin{lstlisting}[language=Python,caption=素数判定プログラム(直接記述の場合),label=src:PrimeNumDescription]
# 素数か否かを判定するプログラム

num = 29

# ユーザが入力する場合
#num = int(input("Enter a number: "))

# フラグ変数の定義
flag = False

if num == 1:
    print(num, "is not a prime number")
elif num > 1:
    # 約数のチェック
    for i in range(2, num):
        if (num % i) == 0:
            # 約数が見つかったらflagをTrueにする
            flag = True
            # ループを脱ける
            break

    # flagがTrueかチェック
    if flag:
        print(num, "is not a prime number")
    else:
        print(num, "is a prime number")
\end{lstlisting}


\subsection{ソースファイルを読み込む方法}
\label{subsec:lstinputlisting}

ソースファイルを{\TeX}ファイルに読み込んで掲載する場合は,\texttt{lstinputlisting}コマンドを用いる.
\texttt{lstinputlisting}のオプションは\subsecref{subsec:lstlisting}に示した\texttt{lstlisting}環境のオプションと同じである.

\begin{code}
\lstinputlisting[language=プログラミング言語,caption=キャプション,label=src:ラベル]{ソースファイルパス}
\end{code}%

素数を判定するPythonプログラムのソースファイル\texttt{src/primenum.py}の読み込み例を\srcref{src:PrimeNumFile}に示す.

\lstinputlisting[language=Python,caption=素数判定プログラム(ソースファイル読み込みの場合),label=src:PrimeNumFile]{src/primenum.py}



\chapter{使用しているパッケージと追加方法}\label{apdx:package}

本{\TeX}文章が採用しているスタイルファイル\texttt{UCLabThesis.sty}には,必要なスタイルファイルを利用するための記述がある.
各自の環境に必要なスタイルファイルがない場合は,下記の手順でスタイルを適用してほしい.
\begin{enumerate}
    \item インターネットから利用したいスタイルファイルを入手する.
    \item 各自のOverleaf環境にダウンロードしたスタイルファイルをアップロードする.
    \item \texttt{thesis.tex}で\texttt{UCLabThesis.sty}を読み込んでいる行の下に\verb|\usepackage{...}|を使ってスタイルファイルを適用する.
\end{enumerate}

\endinput