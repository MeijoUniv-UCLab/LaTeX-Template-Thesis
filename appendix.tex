% 鈴木研究室 卒業論文付録テンプレート
% Version: 1.1
% Last Update: 2011/12/31 by Hidekazu Suzuki
% Contact to: hsuzuki@meijo-u.ac.jp
% Character code: UTF-8

\chapter{付録に掲載する内容例}\label{apdx:Example}

付録には本文の内容を補う情報,データなどを掲載するとよい.
例えば,以下のような内容が付録として適切である.

\begin{itemize}
\item プロトコルの仕様,シーケンス,パケットフォーマットなど
\item プログラムのフローチャート,アルゴリズムなど
\item 実装の詳細な情報,インストール方法,実行方法など
\item 関連研究の詳細
\item 本文中に記載しなかった実験データおよび評価結果など
\item 本文中で使用する記号の定義
\item 研究の過程で得られた知見,修正したプログラムのバグなど
\end{itemize}


\chapter{使用しているパッケージ}\label{apdx:package}

本{\TeX}文章が採用しているスタイルファイル\texttt{UCLabThesis.sty}は,下記のパッケージを利用している.
各自の環境に必要なスタイルファイルがない場合は,インターネットから入手してインストールする必要がある.

\begin{itemize}
\item \texttt{geometry}
\item \texttt{times}
\item \texttt{mathptmx}
\item \texttt{amsmath}
\item \texttt{amssymb}
\item \texttt{extarrows}
\item \texttt{esvect}
\item \texttt{graphicx}
\item \texttt{mediabb}
\item \texttt{caption}
\item \texttt{subfig}
\item \texttt{array}
\item \texttt{multirow}
\item \texttt{fancybox}
\item \texttt{ascmac}
\item \texttt{framed}
\item \texttt{eclbkbox}
\item \texttt{enumerate}
\item \texttt{enumitem}
\item \texttt{cite}
\item \texttt{url}
\end{itemize}

\endinput